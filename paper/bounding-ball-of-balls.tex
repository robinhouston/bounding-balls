\documentclass[a4paper]{article}
\usepackage[round]{natbib}

\bibliographystyle{plainnat}
\frenchspacing

\newcommand\R{\mathbb{R}}

\author{Mike Bostock and Robin Houston}
\title{Computing the smallest enclosing ball of balls}

\begin{document}
\maketitle

\abstract{We consider the problem of computing the smallest ball enclosing a set
of balls. There is an elegant randomised algorithm due to \citet{Welzl} that
computes the smallest ball enclosing a set of \emph{points}: it was shown by
\citet{Fischer-Gartner, FischerThesis} that a natural extension of Welzl's algorithm
to sets of balls does not work in general. The aim of this note is to point out that
a trivial modification to this algorithm makes it work correctly.}

\section{Welzl's algorithm}

\section{}

\section{Proof of correctness}

\section{Practical considerations}
[Just like Welzl’s original algorithm. We can just shuffle the input balls once at the outset, rather than making a random choice at each step. This is more efficient and still runs in expected linear time. Also the move-to-front heuristic seems to make it faster in practice: explain why this is intuitively plausible, though not supported by rigorous arguments.]

\section{Move-to-front}
[Can we say anything about whether the move-to-front heuristic has the effect that recursive calls are always tight?]

\bibliography{refs}
\end{document}
